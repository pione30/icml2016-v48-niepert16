% !TeX encoding = UTF-8
% !TeX program = upLaTeX + dvipdfmx

\documentclass[dvipdfmx]{beamer}
\mode<presentation>
\usepackage{bxdpx-beamer}  % ナビゲーションシンボルを機能させる
\usepackage{pxjahyper}     % しおりの文字化け対策
\renewcommand{\kanjifamilydefault}{\gtdefault} % 和文既定をゴシックに変更

\usepackage{tikz}
\usetikzlibrary{positioning}

\usetheme{Frankfurt}

\setbeamertemplate{navigation symbols}{}   % ナビゲーションシンボルを消す
\setbeamertemplate{footline}[frame number] % フッターをスライド番号だけにする
\setbeamertemplate{blocks}[rounded][shadow=false]

\title{Example Presentation Created with the Beamer Package}
\author{pione30}
\date{\today}

\begin{document}

  \frame[plain]{\titlepage}

\section*{Outline}

  \begin{frame}
    \tableofcontents
  \end{frame}

\section{Theorem and proof}

  \begin{frame}
    \frametitle{There Is No Largest Prime Number}
    \framesubtitle{The proof uses \textit{reductio ad absurdum}.}
    
    \begin{theorem}
      There is no largest prime number.
    \end{theorem}

    \begin{proof}
      \begin{enumerate}
        \item<1-| alert@1> Suppose $p$ were the largest prime number.
        \item<2-> Let $q$ be the product of the first $p$ numbers.
        \item<3-> Then $q+1$ is not divisible by any of them.
        \item<4-> But $q + 1$ is greater than $1$, thus divisible by some prime
        number not in the first $p$ numbers.\qedhere
      \end{enumerate}
    \end{proof}
  \end{frame}

\section{日本語のテスト}

  \begin{frame}{日本語のテスト}
    \begin{block}{日本語ブロック}
      テストだよ

      いろはにほへとちりぬるを\\
      \visible<2->{\alert{わかよたれそつねならむ}}
    \end{block}
  \end{frame}

\end{document}
